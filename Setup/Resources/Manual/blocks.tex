\subsection{Air}\label{subsec:blocks_air}
The air block that fills the world. Could also be interpreted as "no block".
\newline
\begin{itemize}[nosep]
    \item[ID:] \texttt{Air}
    \item[Solid:]  \XSolidBrush \item[Interactions:]  \XSolidBrush \item[Replaceable:]  \Checkmark
\end{itemize}

\subsection{Grass}\label{subsec:blocks_grass}
Dirt with some grass on top. Plants can be placed on top of this.
The grass can burn, creating ash.
\newline
\begin{itemize}[nosep]
    \item[ID:] \texttt{Grass}
    \item[Solid:]  \Checkmark \item[Interactions:]  \XSolidBrush \item[Replaceable:]  \XSolidBrush
\end{itemize}

\subsection{Ash-covered Dirt}\label{subsec:blocks_ash-covered dirt}
Grass that was burned. Water can burn the ash away.
\newline
\begin{itemize}[nosep]
    \item[ID:] \texttt{GrassBurned}
    \item[Solid:]  \Checkmark \item[Interactions:]  \XSolidBrush \item[Replaceable:]  \XSolidBrush
\end{itemize}

\subsection{Dirt}\label{subsec:blocks_dirt}
Simple dirt. Grass next to it can spread over it.
\newline
\begin{itemize}[nosep]
    \item[ID:] \texttt{Dirt}
    \item[Solid:]  \Checkmark \item[Interactions:]  \XSolidBrush \item[Replaceable:]  \XSolidBrush
\end{itemize}

\subsection{Farmland}\label{subsec:blocks_farmland}
Tilled dirt that allows many plants to grow.
While plants can also grow on normal grass, this block allows full growth.
\newline
\begin{itemize}[nosep]
    \item[ID:] \texttt{Farmland}
    \item[Solid:]  \Checkmark \item[Interactions:]  \XSolidBrush \item[Replaceable:]  \XSolidBrush
\end{itemize}

\subsection{Tall Grass}\label{subsec:blocks_tall grass}
A tall grassy plant. Fluids will destroy it, if the level is too high.
\newline
\begin{itemize}[nosep]
    \item[ID:] \texttt{TallGrass}
    \item[Solid:]  \XSolidBrush \item[Interactions:]  \XSolidBrush \item[Replaceable:]  \Checkmark
\end{itemize}

\subsection{Very Tall Grass}\label{subsec:blocks_very tall grass}
A much larger version of the normal tall grass.
\newline
\begin{itemize}[nosep]
    \item[ID:] \texttt{VeryTallGrass}
    \item[Solid:]  \XSolidBrush \item[Interactions:]  \XSolidBrush \item[Replaceable:]  \XSolidBrush
\end{itemize}

\subsection{Flower}\label{subsec:blocks_flower}
A simple flower.
\newline
\begin{itemize}[nosep]
    \item[ID:] \texttt{Flower}
    \item[Solid:]  \XSolidBrush \item[Interactions:]  \XSolidBrush \item[Replaceable:]  \Checkmark
\end{itemize}

\subsection{Tall Flower}\label{subsec:blocks_tall flower}
A very tall flower.
\newline
\begin{itemize}[nosep]
    \item[ID:] \texttt{TallFlower}
    \item[Solid:]  \XSolidBrush \item[Interactions:]  \XSolidBrush \item[Replaceable:]  \XSolidBrush
\end{itemize}

\subsection{Stone}\label{subsec:blocks_stone}
This stone block makes up large parts of the world. Below the dirt layers, the ground is solid stone.
\newline
\begin{itemize}[nosep]
    \item[ID:] \texttt{Stone}
    \item[Solid:]  \Checkmark \item[Interactions:]  \XSolidBrush \item[Replaceable:]  \XSolidBrush
\end{itemize}

\subsection{Rubble}\label{subsec:blocks_rubble}
When stone is destroyed, rubble is what remains.
\newline
\begin{itemize}[nosep]
    \item[ID:] \texttt{Rubble}
    \item[Solid:]  \Checkmark \item[Interactions:]  \XSolidBrush \item[Replaceable:]  \XSolidBrush
\end{itemize}

\subsection{Mud}\label{subsec:blocks_mud}
Mud is created when water and dirt mix.
\newline
\begin{itemize}[nosep]
    \item[ID:] \texttt{Mud}
    \item[Solid:]  \Checkmark \item[Interactions:]  \XSolidBrush \item[Replaceable:]  \XSolidBrush
\end{itemize}

\subsection{Pumice}\label{subsec:blocks_pumice}
Pumice is created when lava rapidly cools down, while being in contact with a lot of water.
\newline
\begin{itemize}[nosep]
    \item[ID:] \texttt{Pumice}
    \item[Solid:]  \Checkmark \item[Interactions:]  \XSolidBrush \item[Replaceable:]  \XSolidBrush
\end{itemize}

\subsection{Obsidian}\label{subsec:blocks_obsidian}
Obsidian is a dark type of stone, that forms from lava.
\newline
\begin{itemize}[nosep]
    \item[ID:] \texttt{Obsidian}
    \item[Solid:]  \Checkmark \item[Interactions:]  \XSolidBrush \item[Replaceable:]  \XSolidBrush
\end{itemize}

\subsection{Snow}\label{subsec:blocks_snow}
Snow covers the ground, and can have different heights.
\newline
\begin{itemize}[nosep]
    \item[ID:] \texttt{Snow}
    \item[Solid:]  \Checkmark \item[Interactions:]  \Checkmark \item[Replaceable:]  \XSolidBrush
\end{itemize}

\subsection{Leaves}\label{subsec:blocks_leaves}
Leaves are transparent parts of the tree. They are flammable.
\newline
\begin{itemize}[nosep]
    \item[ID:] \texttt{Leaves}
    \item[Solid:]  \Checkmark \item[Interactions:]  \XSolidBrush \item[Replaceable:]  \XSolidBrush
\end{itemize}

\subsection{Log}\label{subsec:blocks_log}
Log is the unprocessed, wooden part of a tree. As it is made of wood, it is flammable.
\newline
\begin{itemize}[nosep]
    \item[ID:] \texttt{Log}
    \item[Solid:]  \Checkmark \item[Interactions:]  \XSolidBrush \item[Replaceable:]  \XSolidBrush
\end{itemize}

\subsection{Wood}\label{subsec:blocks_wood}
Processed wood that can be used as construction material. It is flammable.
\newline
\begin{itemize}[nosep]
    \item[ID:] \texttt{Wood}
    \item[Solid:]  \Checkmark \item[Interactions:]  \XSolidBrush \item[Replaceable:]  \XSolidBrush
\end{itemize}

\subsection{Sand}\label{subsec:blocks_sand}
Sand naturally forms and allows water to flow through it.
\newline
\begin{itemize}[nosep]
    \item[ID:] \texttt{Sand}
    \item[Solid:]  \Checkmark \item[Interactions:]  \XSolidBrush \item[Replaceable:]  \XSolidBrush
\end{itemize}

\subsection{Gravel}\label{subsec:blocks_gravel}
Gravel, which is made out of small pebbles, allows water to flow through it.
\newline
\begin{itemize}[nosep]
    \item[ID:] \texttt{Gravel}
    \item[Solid:]  \Checkmark \item[Interactions:]  \XSolidBrush \item[Replaceable:]  \XSolidBrush
\end{itemize}

\subsection{Coal Ore}\label{subsec:blocks_coal ore}
Coal ore is stone that contains coal.
\newline
\begin{itemize}[nosep]
    \item[ID:] \texttt{OreCoal}
    \item[Solid:]  \Checkmark \item[Interactions:]  \XSolidBrush \item[Replaceable:]  \XSolidBrush
\end{itemize}

\subsection{Iron Ore}\label{subsec:blocks_iron ore}
Iron ore is stone that contains iron.
\newline
\begin{itemize}[nosep]
    \item[ID:] \texttt{OreIron}
    \item[Solid:]  \Checkmark \item[Interactions:]  \XSolidBrush \item[Replaceable:]  \XSolidBrush
\end{itemize}

\subsection{Gold Ore}\label{subsec:blocks_gold ore}
Gold ore is stone that contains gold.
\newline
\begin{itemize}[nosep]
    \item[ID:] \texttt{OreGold}
    \item[Solid:]  \Checkmark \item[Interactions:]  \XSolidBrush \item[Replaceable:]  \XSolidBrush
\end{itemize}

\subsection{Ash}\label{subsec:blocks_ash}
Ahs is the remainder of burning processes.
\newline
\begin{itemize}[nosep]
    \item[ID:] \texttt{Ash}
    \item[Solid:]  \Checkmark \item[Interactions:]  \XSolidBrush \item[Replaceable:]  \XSolidBrush
\end{itemize}

\subsection{Cactus}\label{subsec:blocks_cactus}
A cactus slowly grows upwards. It can only be placed on sand.
\newline
\begin{itemize}[nosep]
    \item[ID:] \texttt{Cactus}
    \item[Solid:]  \Checkmark \item[Interactions:]  \XSolidBrush \item[Replaceable:]  \XSolidBrush
\end{itemize}

\subsection{Pumpkin}\label{subsec:blocks_pumpkin}
Pumpkins are the fruit of the pumpkin plant. They have to be placed on solid ground.
\newline
\begin{itemize}[nosep]
    \item[ID:] \texttt{Pumpkin}
    \item[Solid:]  \Checkmark \item[Interactions:]  \XSolidBrush \item[Replaceable:]  \XSolidBrush
\end{itemize}

\subsection{Melon}\label{subsec:blocks_melon}
Melons are the fruit of the melon plant. They have to be placed on solid ground.
\newline
\begin{itemize}[nosep]
    \item[ID:] \texttt{Melon}
    \item[Solid:]  \Checkmark \item[Interactions:]  \XSolidBrush \item[Replaceable:]  \XSolidBrush
\end{itemize}

\subsection{Spider Web}\label{subsec:blocks_spider web}
Spiderwebs slow the movement of entities and can be used to trap enemies.
\newline
\begin{itemize}[nosep]
    \item[ID:] \texttt{Spiderweb}
    \item[Solid:]  \XSolidBrush \item[Interactions:]  \XSolidBrush \item[Replaceable:]  \XSolidBrush
\end{itemize}

\subsection{Vines}\label{subsec:blocks_vines}
Vines grow downwards, and can hang freely. It is possible to climb them.
\newline
\begin{itemize}[nosep]
    \item[ID:] \texttt{Vines}
    \item[Solid:]  \XSolidBrush \item[Interactions:]  \XSolidBrush \item[Replaceable:]  \XSolidBrush
\end{itemize}

\subsection{Flax}\label{subsec:blocks_flax}
Flax is a crop plant that grows on farmland. It requires water to fully grow.
\newline
\begin{itemize}[nosep]
    \item[ID:] \texttt{Flax}
    \item[Solid:]  \XSolidBrush \item[Interactions:]  \XSolidBrush \item[Replaceable:]  \XSolidBrush
\end{itemize}

\subsection{Potatoes}\label{subsec:blocks_potatoes}
Potatoes are a crop plant that grows on farmland. They requires water to fully grow.
\newline
\begin{itemize}[nosep]
    \item[ID:] \texttt{Potatoes}
    \item[Solid:]  \XSolidBrush \item[Interactions:]  \XSolidBrush \item[Replaceable:]  \XSolidBrush
\end{itemize}

\subsection{Onions}\label{subsec:blocks_onions}
Onions are a crop plant that grows on farmland. They requires water to fully grow.
\newline
\begin{itemize}[nosep]
    \item[ID:] \texttt{Onions}
    \item[Solid:]  \XSolidBrush \item[Interactions:]  \XSolidBrush \item[Replaceable:]  \XSolidBrush
\end{itemize}

\subsection{Wheat}\label{subsec:blocks_wheat}
Wheat is a crop plant that grows on farmland. It requires water to fully grow.
\newline
\begin{itemize}[nosep]
    \item[ID:] \texttt{Wheat}
    \item[Solid:]  \XSolidBrush \item[Interactions:]  \XSolidBrush \item[Replaceable:]  \XSolidBrush
\end{itemize}

\subsection{Maize}\label{subsec:blocks_maize}
Maize is a crop plant that grows on farmland.
Maize grows two blocks high. It requires water to fully grow.
\newline
\begin{itemize}[nosep]
    \item[ID:] \texttt{Maize}
    \item[Solid:]  \XSolidBrush \item[Interactions:]  \XSolidBrush \item[Replaceable:]  \XSolidBrush
\end{itemize}

\subsection{Pumpkin Plant}\label{subsec:blocks_pumpkin plant}
The pumpkin plant grows pumpkin fruits.
\newline
\begin{itemize}[nosep]
    \item[ID:] \texttt{PumpkinPlant}
    \item[Solid:]  \XSolidBrush \item[Interactions:]  \XSolidBrush \item[Replaceable:]  \XSolidBrush
\end{itemize}

\subsection{Melon Plant}\label{subsec:blocks_melon plant}
The melon plant grows melon fruits.
\newline
\begin{itemize}[nosep]
    \item[ID:] \texttt{MelonPlant}
    \item[Solid:]  \XSolidBrush \item[Interactions:]  \XSolidBrush \item[Replaceable:]  \XSolidBrush
\end{itemize}

\subsection{Glass}\label{subsec:blocks_glass}
Glass is transparent block.
\newline
\begin{itemize}[nosep]
    \item[ID:] \texttt{Glass}
    \item[Solid:]  \Checkmark \item[Interactions:]  \XSolidBrush \item[Replaceable:]  \XSolidBrush
\end{itemize}

\subsection{Tiled Glass}\label{subsec:blocks_tiled glass}
Tiled glass is like glass, but made out of four tiles.
\newline
\begin{itemize}[nosep]
    \item[ID:] \texttt{GlassTiled}
    \item[Solid:]  \Checkmark \item[Interactions:]  \XSolidBrush \item[Replaceable:]  \XSolidBrush
\end{itemize}

\subsection{Steel}\label{subsec:blocks_steel}
The steel block is a metal construction block.
\newline
\begin{itemize}[nosep]
    \item[ID:] \texttt{Steel}
    \item[Solid:]  \Checkmark \item[Interactions:]  \XSolidBrush \item[Replaceable:]  \XSolidBrush
\end{itemize}

\subsection{Worked Stone}\label{subsec:blocks_worked stone}
Worked stone is a processed stone block.
\newline
\begin{itemize}[nosep]
    \item[ID:] \texttt{StoneWorked}
    \item[Solid:]  \Checkmark \item[Interactions:]  \XSolidBrush \item[Replaceable:]  \XSolidBrush
\end{itemize}

\subsection{Ladder}\label{subsec:blocks_ladder}
A ladder allows climbing up and down.
\newline
\begin{itemize}[nosep]
    \item[ID:] \texttt{Ladder}
    \item[Solid:]  \XSolidBrush \item[Interactions:]  \XSolidBrush \item[Replaceable:]  \XSolidBrush
\end{itemize}

\subsection{Small Tiles}\label{subsec:blocks_small tiles}
Small tiles for construction of floors and walls.
\newline
\begin{itemize}[nosep]
    \item[ID:] \texttt{TilesSmall}
    \item[Solid:]  \Checkmark \item[Interactions:]  \XSolidBrush \item[Replaceable:]  \XSolidBrush
\end{itemize}

\subsection{Large Tiles}\label{subsec:blocks_large tiles}
Large tiles for construction of floors and walls.
\newline
\begin{itemize}[nosep]
    \item[ID:] \texttt{TilesLarge}
    \item[Solid:]  \Checkmark \item[Interactions:]  \XSolidBrush \item[Replaceable:]  \XSolidBrush
\end{itemize}

\subsection{Black Checkerboard Tiles}\label{subsec:blocks_black checkerboard tiles}
Black checkerboard tiles come in different colors.
\newline
\begin{itemize}[nosep]
    \item[ID:] \texttt{TilesCheckerboardBlack}
    \item[Solid:]  \Checkmark \item[Interactions:]  \Checkmark \item[Replaceable:]  \XSolidBrush
\end{itemize}

\subsection{White Checkerboard Tiles}\label{subsec:blocks_white checkerboard tiles}
White checkerboard tiles come in different colors.
\newline
\begin{itemize}[nosep]
    \item[ID:] \texttt{TilesCheckerboardWhite}
    \item[Solid:]  \Checkmark \item[Interactions:]  \Checkmark \item[Replaceable:]  \XSolidBrush
\end{itemize}

\subsection{Bricks}\label{subsec:blocks_bricks}
Bricks are a simple construction material.
\newline
\begin{itemize}[nosep]
    \item[ID:] \texttt{Bricks}
    \item[Solid:]  \Checkmark \item[Interactions:]  \XSolidBrush \item[Replaceable:]  \XSolidBrush
\end{itemize}

\subsection{Paving Stone}\label{subsec:blocks_paving stone}
Paving stone is a simple construction material, ideal for paths.
\newline
\begin{itemize}[nosep]
    \item[ID:] \texttt{PavingStone}
    \item[Solid:]  \Checkmark \item[Interactions:]  \XSolidBrush \item[Replaceable:]  \XSolidBrush
\end{itemize}

\subsection{Red Plastic}\label{subsec:blocks_red plastic}
Red plastic is a construction material.
\newline
\begin{itemize}[nosep]
    \item[ID:] \texttt{RedPlastic}
    \item[Solid:]  \Checkmark \item[Interactions:]  \XSolidBrush \item[Replaceable:]  \XSolidBrush
\end{itemize}

\subsection{Concrete}\label{subsec:blocks_concrete}
Concrete is a flexible construction material that can have different heights and colors.
It can be build using fluid concrete.
\newline
\begin{itemize}[nosep]
    \item[ID:] \texttt{Concrete}
    \item[Solid:]  \Checkmark \item[Interactions:]  \Checkmark \item[Replaceable:]  \XSolidBrush
\end{itemize}

\subsection{Stone Face}\label{subsec:blocks_stone face}
This block is like a processed stone block, but with a decorative face added.
\newline
\begin{itemize}[nosep]
    \item[ID:] \texttt{StoneFace}
    \item[Solid:]  \Checkmark \item[Interactions:]  \XSolidBrush \item[Replaceable:]  \XSolidBrush
\end{itemize}

\subsection{Vase}\label{subsec:blocks_vase}
The vase is a decorative block that must be placed on solid ground.
\newline
\begin{itemize}[nosep]
    \item[ID:] \texttt{Vase}
    \item[Solid:]  \Checkmark \item[Interactions:]  \XSolidBrush \item[Replaceable:]  \XSolidBrush
\end{itemize}

\subsection{Bed}\label{subsec:blocks_bed}
The bed can be placed to set a different spawn point.
It is possible to change to color of a bed.
\newline
\begin{itemize}[nosep]
    \item[ID:] \texttt{Bed}
    \item[Solid:]  \Checkmark \item[Interactions:]  \Checkmark \item[Replaceable:]  \XSolidBrush
\end{itemize}

\subsection{Wool}\label{subsec:blocks_wool}
Wool is a flammable material, that allows its color to be changed.
\newline
\begin{itemize}[nosep]
    \item[ID:] \texttt{Wool}
    \item[Solid:]  \Checkmark \item[Interactions:]  \Checkmark \item[Replaceable:]  \XSolidBrush
\end{itemize}

\subsection{Decorated Wool}\label{subsec:blocks_decorated wool}
Decorated wool is similar to wool, decorated with golden ornaments.
\newline
\begin{itemize}[nosep]
    \item[ID:] \texttt{WoolDecorated}
    \item[Solid:]  \Checkmark \item[Interactions:]  \Checkmark \item[Replaceable:]  \XSolidBrush
\end{itemize}

\subsection{Carpet}\label{subsec:blocks_carpet}
Carpets can be used to cover the floor. Their color can be changed.
\newline
\begin{itemize}[nosep]
    \item[ID:] \texttt{Carpet}
    \item[Solid:]  \Checkmark \item[Interactions:]  \Checkmark \item[Replaceable:]  \XSolidBrush
\end{itemize}

\subsection{Decorated Carpet}\label{subsec:blocks_decorated carpet}
Decorated carpets are similar to carpets, decorated with golden ornaments.
\newline
\begin{itemize}[nosep]
    \item[ID:] \texttt{CarpetDecorated}
    \item[Solid:]  \Checkmark \item[Interactions:]  \Checkmark \item[Replaceable:]  \XSolidBrush
\end{itemize}

\subsection{Glass Pane}\label{subsec:blocks_glass pane}
Glass panes are a thin alternative to glass blocks.
They connect to some neighboring blocks.
\newline
\begin{itemize}[nosep]
    \item[ID:] \texttt{GlassPane}
    \item[Solid:]  \Checkmark \item[Interactions:]  \XSolidBrush \item[Replaceable:]  \XSolidBrush
\end{itemize}

\subsection{Bars}\label{subsec:blocks_bars}
Steel bars are a thin, but strong barrier.
\newline
\begin{itemize}[nosep]
    \item[ID:] \texttt{Bars}
    \item[Solid:]  \Checkmark \item[Interactions:]  \XSolidBrush \item[Replaceable:]  \XSolidBrush
\end{itemize}

\subsection{Wooden Fence}\label{subsec:blocks_wooden fence}
The wooden fence can be used as way of marking areas. It does not prevent jumping over it.
As this fence is made out of wood, it is flammable. Fences can connect to other blocks.
\newline
\begin{itemize}[nosep]
    \item[ID:] \texttt{FenceWood}
    \item[Solid:]  \Checkmark \item[Interactions:]  \XSolidBrush \item[Replaceable:]  \XSolidBrush
\end{itemize}

\subsection{Rubble Wall}\label{subsec:blocks_rubble wall}
The rubble wall is a stone barrier that can be used as a way of marking areas.
They do not prevent jumping over it, and can connect to other blocks.
\newline
\begin{itemize}[nosep]
    \item[ID:] \texttt{WallRubble}
    \item[Solid:]  \Checkmark \item[Interactions:]  \XSolidBrush \item[Replaceable:]  \XSolidBrush
\end{itemize}

\subsection{Brick Wall}\label{subsec:blocks_brick wall}
The brick wall is similar to all other walls, and made out of bricks.
They do not prevent jumping over them, and can connect to other blocks.
\newline
\begin{itemize}[nosep]
    \item[ID:] \texttt{WallBricks}
    \item[Solid:]  \Checkmark \item[Interactions:]  \XSolidBrush \item[Replaceable:]  \XSolidBrush
\end{itemize}

\subsection{Steel Door}\label{subsec:blocks_steel door}
The steel door allows closing of a room. It can be opened and closed.
\newline
\begin{itemize}[nosep]
    \item[ID:] \texttt{DoorSteel}
    \item[Solid:]  \Checkmark \item[Interactions:]  \Checkmark \item[Replaceable:]  \XSolidBrush
\end{itemize}

\subsection{Wooden Door}\label{subsec:blocks_wooden door}
The wooden door allows closing of a room. It can be opened and closed.
As this door is made out of wood, it is flammable.
\newline
\begin{itemize}[nosep]
    \item[ID:] \texttt{DoorWood}
    \item[Solid:]  \Checkmark \item[Interactions:]  \Checkmark \item[Replaceable:]  \XSolidBrush
\end{itemize}

\subsection{Wooden Gate}\label{subsec:blocks_wooden gate}
Fence gates are meant as a passage trough fences and walls.
\newline
\begin{itemize}[nosep]
    \item[ID:] \texttt{GateWood}
    \item[Solid:]  \Checkmark \item[Interactions:]  \Checkmark \item[Replaceable:]  \XSolidBrush
\end{itemize}

\subsection{Barrier}\label{subsec:blocks_barrier}
The fluid barrier can be used to control fluid flow. It can be opened and closed.
It does not prevent gasses from flowing through it.
\newline
\begin{itemize}[nosep]
    \item[ID:] \texttt{FluidBarrier}
    \item[Solid:]  \Checkmark \item[Interactions:]  \Checkmark \item[Replaceable:]  \XSolidBrush
\end{itemize}

\subsection{Steel Pipe}\label{subsec:blocks_steel pipe}
The industrial steel pipe can be used to control fluid flow.
It connects to other pipes.
\newline
\begin{itemize}[nosep]
    \item[ID:] \texttt{SteelPipe}
    \item[Solid:]  \Checkmark \item[Interactions:]  \XSolidBrush \item[Replaceable:]  \XSolidBrush
\end{itemize}

\subsection{Wooden Pipe}\label{subsec:blocks_wooden pipe}
The wooden pipe offers a primitive way of controlling fluid flow.
It connects to other pipes.
\newline
\begin{itemize}[nosep]
    \item[ID:] \texttt{WoodenPipe}
    \item[Solid:]  \Checkmark \item[Interactions:]  \XSolidBrush \item[Replaceable:]  \XSolidBrush
\end{itemize}

\subsection{Straight Steelpipe}\label{subsec:blocks_straight steelpipe}
This pipe is a special steel pipe that can only form straight connections.
It is ideal for parallel pipes.
\newline
\begin{itemize}[nosep]
    \item[ID:] \texttt{StraightSteelPipe}
    \item[Solid:]  \Checkmark \item[Interactions:]  \XSolidBrush \item[Replaceable:]  \XSolidBrush
\end{itemize}

\subsection{Valve Pipe}\label{subsec:blocks_valve pipe}
This is a special steel pipe that can be closed. It prevents all fluid flow.
\newline
\begin{itemize}[nosep]
    \item[ID:] \texttt{PipeValve}
    \item[Solid:]  \Checkmark \item[Interactions:]  \Checkmark \item[Replaceable:]  \XSolidBrush
\end{itemize}

\subsection{Pump}\label{subsec:blocks_pump}
The pump can lift fluids up when interacted with.
It can only lift up to a threshold of 16 blocks.
\newline
\begin{itemize}[nosep]
    \item[ID:] \texttt{Pump}
    \item[Solid:]  \Checkmark \item[Interactions:]  \Checkmark \item[Replaceable:]  \XSolidBrush
\end{itemize}

\subsection{Fire}\label{subsec:blocks_fire}
Fire is a dangerous block that spreads onto nearby flammable blocks.
When spreading, fire burns blocks which can destroy them.
\newline
\begin{itemize}[nosep]
    \item[ID:] \texttt{Fire}
    \item[Solid:]  \XSolidBrush \item[Interactions:]  \XSolidBrush \item[Replaceable:]  \Checkmark
\end{itemize}

\subsection{Pulsating Block}\label{subsec:blocks_pulsating block}
This is a magical pulsating block.
\newline
\begin{itemize}[nosep]
    \item[ID:] \texttt{Pulsating}
    \item[Solid:]  \Checkmark \item[Interactions:]  \Checkmark \item[Replaceable:]  \XSolidBrush
\end{itemize}

\subsection{Eternal Flame}\label{subsec:blocks_eternal flame}
The eternal flame, once lit, will never go out naturally.
\newline
\begin{itemize}[nosep]
    \item[ID:] \texttt{EternalFlame}
    \item[Solid:]  \Checkmark \item[Interactions:]  \XSolidBrush \item[Replaceable:]  \XSolidBrush
\end{itemize}

\subsection{Path}\label{subsec:blocks_path}
The path is a dirt block with its top layer trampled.
\newline
\begin{itemize}[nosep]
    \item[ID:] \texttt{Path}
    \item[Solid:]  \Checkmark \item[Interactions:]  \XSolidBrush \item[Replaceable:]  \XSolidBrush
\end{itemize}
